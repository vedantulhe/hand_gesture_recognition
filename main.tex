

\documentclass[biblatex]{lni}
\addbibresource{lni-paper-example-en.bib}


\usepackage{booktabs}
\usepackage[]{blindtext}

\begin{document}

\title[A Short Title]{Human Activity Recognition}

\author[1]{Firstname1 Lastname1}{firstname1.lastname1@affiliation1.de}{0000-0000-0000-0000}
\author[2]{Firstname2 Lastname2}{firstname2.lastname2@affiliation2.de}{0000-0000-0000-0000}
\author[1]{Firstname3 Lastname3}{firstname3.lastname3@affiliation1.de}{0000-0000-0000-0000}
\author[1]{Firstname4 Lastname4}{firstname4.lastname4@affiliation1.de}{0000-0000-0000-0000}%
\affil[1]{University\\Department\\Street\\Postal Code City\\Country}
\affil[2]{University\\Department\\Address\\Country}
\maketitle

\begin{abstract}
The \LaTeX{} class \texttt{lni} implements the layout guidelines for contributions in LNI conference volumes.
This document describes its usage and serves as an example for its presentation.
The abstract is a brief overview of the work, which should be between 70 and 150 words long and should contain the most important information.
The formatting is done automatically within the abstract environment.
\end{abstract}

\begin{keywords}
LNI Guidelines \and \LaTeX\ Template
\end{keywords}

\section{Usage}
The GI provides guidelines for formatting documents in the LNI series at \url{http://www.gi-ev.de/LNI}.
For \LaTeX{} documents, these are realized through the document class \texttt{lni}.

This document is based on the official documentation, simplified, and assumes basic LaTeX knowledge.
Generic placeholders are inserted at the appropriate locations (such as the author information) and are not further documented elsewhere.

This template is structured as follows:
\Cref{sec:demos} demonstrates the LNI features.
\Cref{sec:lniconformance} demonstrates compliance with the guidelines through simple text.

\section{Demonstrations}
\label{sec:demos}
The power set symbol ($\powerset$) is displayed correctly.
There is no more Weierstrass-p ($\wp$).

Angle brackets can be entered directly: <test />

Anonymizations can be made automatically using the anonymous option in the document class. For this purpose, there is the \texttt{anon} macro, e.g., \anon{Secret for Review} and \anon[for Review Only]{for the final version only}.

Here's a small demonstration of \href{https://www.ctan.org/pkg/microtype}{microtype}:
\blindtext

\section{Demonstration of Compliance with the Guidelines}
\label{sec:lniconformance}

\subsection{Bibliography}
The last section shows an exemplary bibliography for books with one author \cite{Ez10} and two authors \cite{AB00}, a contribution in proceedings with three authors \cite{ABC01}, a contribution in an LNI volume with more than three authors \cite{Az09}, two books with the same four authors in the same year \cite{Wa14} and \cite{Wa14b}, a journal \cite{Gl06}, a website \cite{GI19}, and other literature without specific authorship \cite{XX14}.
Biblatex is used because it supports UTF8 cleanly and does not produce errors when bibtexing, \href{https://github.com/gi-ev/LNI/issues/5}{unlike lni.bst}.

References should not be directly embedded as subjects, but only through author references:
Example: \Citet{AB00} provide an example, but also \citet{Az09}.
Note: Capital C with \texttt{Citet} when it begins a sentence. This is analogous to \texttt{Cref}.

Formatting and abbreviations are automatically applied for the reference types \texttt{book}, \texttt{inbook}, \texttt{proceedings}, \texttt{inproceedings}, \texttt{article}, \texttt{online}, and \texttt{misc}.
Possible fields for references can be found in the sample file \texttt{lni-paper-example-en.bib}.
Other references and fields may need to be adjusted afterwards.

\subsection{Figures}
\Cref{fig:demo} shows a figure.

\begin{figure}
  \centering
  \includegraphics[width=.8\textwidth]{example-image}
  \caption{Demographics}
  \label{fig:demo}
\end{figure}

\subsection{Tables}
\Cref{tab:demo} shows a table.

\begin{table}
\centering
\begin{tabular}{lll}
\toprule
Heading Levels & Example & Font Size and Style \\
\midrule
Title (left-aligned) & The Title \ldots & 14 pt, Bold\\
Heading 1 & 1 Introduction & 12 pt, Bold\\
Heading 2 & 2.1 Title & 10 pt, Bold\\
\bottomrule
\end{tabular}
\caption{The Heading Styles}
\label{tab:demo}
\end{table}

\subsection{Program Code}
The LNI style requires listings to be indented from the left margin.
In the \texttt{lni} document class, this is implemented for the \texttt{verbatim} environment.

\begin{verbatim}
public class Hello {
    public static void main (String[] args) {
        System.out.println("Hello World!");
    }
}
\end{verbatim}

Alternatively, the \texttt{lstlisting} environment can be used.

\Cref{java-hello-world} shows an example using the \texttt{lstlisting} environment. Another example is \cref{python-hello-world}.

\begin{lstlisting}[caption={A Java Program}, label=java-hello-world, language=Java]
public class Hello {
  public static void main (String[] args) {
    System.out.println("Hello World!");
  }
}
\end{lstlisting}

\begin{lstlisting}[caption={A Python Program}, label=python-hello-world, language=Python]
# This program prints Hello, world!

print('Hello, world!')
\end{lstlisting}

\subsection{Formulas and Equations}

Correct indentation and numbering for formulas are ensured with the \texttt{equation} and \texttt{align} environments.

\begin{equation}
  1=4-3
\end{equation}
and
\begin{align}
  2&=7-5\\
  3&=2-1
\end{align}

%% \bibliography{lni-paper-example-en.tex} is not allowed here: biblatex expects this in the preamble
%% Run "biber paper" to generate a bibliography.
\printbibliography

\end{document}
